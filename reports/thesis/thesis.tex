\documentclass[12pt,fleqn,leqno,letterpaper]{article}

\usepackage{gensymb}

\include{preamble}

\title{Luminosity}
\author{
  Cameron, M.
  \and
  Rosales, V.
  \and
  Westermann, J.P.
}
\date{July 30, 2017}

\begin{document}

\maketitle

\newpage
\tableofcontents
\listoffigures
\listoftables

\newpage

Notes (Please Read):
\begin{itemize}
  \item \textit{print(df.to\_latex())} will give a latex export of the dataframe in the jupyter notebook, you can use this to quickly copy paste into latex
  \item \textit{plt.savefig(<path>)} will allow you to save your figure as a png so it can be used in the document/presentation
  \item Make sure to commit and pull as much as possible to avoid merge errors
  \item Don't edit the styles of this document yet, will do that at the end
\end{itemize}

\begin{section}{Introduction}
  \begin{subsection}{Summary}
    TODO Will do this last
  \end{subsection}
  \begin{subsection}{Literature Review}
    \begin{subsubsection}{Luminosity-based Approach}
      TODO Michael: Try to accumulate as much as possible. We have such a long list of papers anyway...\\
      The use of night time light data is increasingly being used in economic papers. The aim of this data is to improve the quality of economic data, especially in many war-torn countries where this data is poor. This data becomes even more difficult to work with once regional economic data is needed. This poor data makes it incredibly difficult to understand the country's economic growth. Light data is seen as a promising development in this area. Not only does it provide information for every country, but it also show the spread of activity throughout each country and region. Throughout this paper light levels are used as a proxy for the GDP of an area. Is it feasible to use this variable as a proxy? \\
      
      The paper \textit{``The Value of Luminosity Data as a Proxy for Economic Statistics''} Chen and Nordhaus (2010) closely examines this assumption. Using the same light images as used in this paper, the paper compares light levels to actual GDP data. The light data is aggregated to a $1\degree\times 1\degree$ are to match the finest economic data available to them, although country level data is also examined. The paper concludes that, although the light data can be quite noisy, it provides a good proxy for GDP. Especially in poor countries with poor economic data. In richer countries with much more detailed statistics, because of this noise in measurements it will not provide improvements to the statistics that have already been collected. \\
      
      These arguments are reiterated in the article \textit{``LIGHTS, CAMERA ... INCOME!''}  Pinkovskiy and Sala-i-Martin (2016) from The Quarterly Journal of Economics. The authors point out the large discrepancy found between national accounts GDP  per  capita and household survey means, both of which are used to study poverty and growth. \\
      
      The correlation 
 		
      
    \end{subsubsection}
    \begin{subsubsection}{Natural Disaster Economics}
      TODO Viviana: obviously, as the expert...
    \end{subsubsection}
  \end{subsection}
\end{section}

\begin{section}{Data}
  \begin{subsection}{Data Description}
    TODO Micheal: Describe what the data looks like, how many observations there are, where we got it, who else has used it etc.
  \end{subsection}
  \begin{subsection}{Data Preprocessing}
    TODO Jonas: Diffing volume issues (size of images)
  \end{subsection}
\end{section}

\begin{section}{Modelling}
  \begin{subsection}{Disaster Impact Models}
    TODO Jonas: Describe other regressions that just study impact of earthquakes and some of the models tried out.
  \end{subsection}
  \begin{subsection}{Panel Model}
    \begin{subsubsection}{Region-based Panel}
      TODO Viviana
    \end{subsubsection}
    \begin{subsubsection}{Section-based Panel}
      TODO Jonas
    \end{subsubsection}
    \begin{subsubsection}{Dynamic Panel}
      TODO Viviana: Describe here how the model that you are using is constructed, where you got it, etc.
    \end{subsubsection}
  \end{subsection}
\end{section}

\begin{section}{Results}
  \begin{subsection}{Case Analysis}
    TODO Micheal: This is where your case analysis for different places goes, try to add some statistical tests etc. if possible. E.g. distribution of light one year vs the next compared to overall time series distribution changes (shocks).
  \end{subsection}
  \begin{subsection}{Modelling Results}
    TODO Viviana: Describe the results of the regression here, significant values and what those values mean.
  \end{subsection}
  \begin{subsection}{Conclusions}
    TODO Will do this just before the summary
  \end{subsection}
  \begin{subsection}{Outlook}
    TODO Jonas
  \end{subsection}
\end{section}

% -- Bibliography (APA style)
\bibliography{references}

\end{document}
